\documentclass{article}
\begin{document}

    \title{様々なゲノムトポロジカル情報によるメチル化状態予測手法の研究}
    \author{Kento Ishikawa and Osamu Maruyama}
    \maketitle
    \newpage

    \begin{abstract}
    \end{abstract}

    \section{はじめに}
    \newpage

    \section{手法}

    \subsection{実験に使用するデータについて}

    \subsection{Zhengらの特徴量の作成}

    \subsection{DNA配列に基づく特徴量の作成}

    \subsection{ATAC-Seqによる特徴量の作成}

    \subsection{Hi-Cによる特徴量の作成}

    \subsection{検定による特徴量削減}

    \subsection{学習モデルの作成}

    \subsection{学習モデルの評価}

    \subsection{検定による実験結果の比較}

\end{document}